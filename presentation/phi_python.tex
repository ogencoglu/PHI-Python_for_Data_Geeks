% Contact: Oguzhan Gencoglu - oguzhan.gencoglu@tut.fi
% LATEX Presentation PHI python tutorial

\documentclass{beamer}

\usepackage{marvosym}
%\usepackage{mhchem}
\usepackage{amsmath}
\usepackage{epstopdf}
%\usepackage{ulem}
\usepackage{cancel}

\DeclareMathSizes{10}{10}{10}{10}

% You can uncomment the themes below if you would like to use a different
% one:
%\usetheme{AnnArbor}
\usetheme{Antibes}
%\usetheme{Bergen}
%\usetheme{Berkeley}
%\usetheme{Berlin}
%\usetheme{Boadilla}
%\usetheme{boxes}
%\usetheme{CambridgeUS}
%\usetheme{Copenhagen}
%\usetheme{Darmstadt}
%\usetheme{default}
%\usetheme{Frankfurt}
%\usetheme{Goettingen}
%\usetheme{Hannover}
%\usetheme{Ilmenau}
%\usetheme{JuanLesPins}
%\usetheme{Luebeck}
%\usetheme{Madrid}
%\usetheme{Malmoe}
%\usetheme{Marburg}
%\usetheme{Montpellier}
%\usetheme{PaloAlto}
%\usetheme{Pittsburgh}
%\usetheme{Rochester}
%\usetheme{Singapore}
%\usetheme{Szeged}
%\usetheme{Warsaw}

\title{PYTHON FOR DATA GEEKS}

% A subtitle is optional and this may be deleted

\author{Ouz Gencoglu}
% - Give the names in the same order as the appear in the paper.
% - Use the \inst{?} command only if the authors have different
%   affiliation.

\institute[TUT] % (optional, but mostly needed)
{

  Tampere University of Technology, Finland

}
% - Use the \inst command only if there are several affiliations.

\date{October 2015}
% - Either use conference name or its abbreviation.

\subject{Data Mining}
% This is only inserted into the PDF information catalog. Can be left
% out. 

% If you have a file called "university-logo-filename.xxx", where xxx
% is a graphic format that can be processed by latex or pdflatex,
% resp., then you can add a logo as follows:

% \pgfdeclareimage[height=0.5cm]{university-logo}{university-logo-filename}
% \logo{\pgfuseimage{university-logo}}

% Delete this, if you do not want the table of contents to pop up at
% the beginning of each subsection:
\AtBeginSubsection[]
{
\begin{frame}<beamer>{Outline}
	\tableofcontents[currentsection,currentsubsection]
\end{frame}
}

% Let's get started
\begin{document}

\begin{frame}
	\titlepage
\end{frame}

\begin{frame}{Outline}
	\tableofcontents
	% You might wish to add the option [pausesections]
\end{frame}

% Section and subsections will appear in the presentation overview
% and table of contents.
\section{Python Basics}

\begin{frame}{Python Basics}{How to start?}

	\begin{itemize}
		\item {Readability is part of the syntax!} \pause
		\item {Indexing starts from 0}\pause
		\item {Cyclic}

	\end{itemize}
	
\end{frame}

\section{Packages}

\begin{frame}{Most Useful Libraries}

	\begin{itemize}
		\item {numpy, scipy, statsmodel, pandas} \pause
		\item {Generic ML: scikit-learn}\pause
		\item {NLP: nltk, spaCy}\pause
		\item {Deep Learning: caffe, keras, lasagne}\pause
		\item {matplotlib, seaborn, bokeh (interactive) }\pause
		\item {PyPy, Cython	}\pause
		\item {rPython, RPy2}
	\end{itemize}
	
\end{frame}

\section{Notes}

\begin{frame}{Notes}

	\begin{itemize}
		\item {use map, zip and lambda} \pause
		\item {do not unzip just read from zip file}\pause
		\item {sklearn pipeline}\pause
		\item {never use gradient boosting from sklearn, instead xgboost}\pause
		\item {set n\_devices = -1 for RF}\pause
		\item {matlab-like (c-like) structs - use dicts}\pause
		\item {use numpy memmap arrays for large files}\pause
		\item {save the list of packages - version control - do not update all the time}\pause
		\item {\url{https://github.com/ogencoglu/Templates/tree/master/Python}}
	\end{itemize}
	
\end{frame}

\begin{frame}{R vs. Python}
	\begin{itemize}

	\item {R was developed by statisticians}\pause
	\item {R has terrible error reporting}\pause
	\item {R syntax is bad: function names do.this, doThis, do\_this}\pause
	\item {Memory and speed issues in R}\pause
	\vspace{0.3cm}
	
	\item {Python has backward compatibility issues between 2.x and 3.x}\pause
	\item {In any case both are slow compared to C++ etc.}\pause
	
	\vspace{0.3cm}
	\textbf{For stats and statistical inference use R. For data cleaning and plotting it does not matter. Anything else: use python. \\} \pause
	\centerline{\textbf{Start learning \textit{Go} and \textit{Julia}}}
	\end{itemize}
\end{frame}

\end{document}


